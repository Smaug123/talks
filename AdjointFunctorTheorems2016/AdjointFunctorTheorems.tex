\documentclass[11pt]{amsart}
\usepackage{geometry}                % See geometry.pdf to learn the layout options. There are lots.
\geometry{a4paper}                   % ... or a4paper or a5paper or ... 
\usepackage{graphicx}
\usepackage{amssymb}
\usepackage{epstopdf}
\usepackage{mdframed}
\usepackage{hyperref}
\usepackage{diagrams}

\newmdtheoremenv{defn}{Definition}
\newmdtheoremenv{thm}{Theorem}

\title{Motivation for the General Adjoint Functor Theorem}
\author{Patrick Stevens}

\begin{document}

\maketitle 
\tiny \begin{center} \url{http://www.patrickstevens.co.uk/misc/AdjointFunctorTheorems/AdjointFunctorTheorems.pdf} \end{center}

\normalsize
\emph{You should draw diagrams yourself throughout this document. It will be unreadable as mere symbols.}

\section{Primeval Adjoint Functor Theorem}

Recall the theorem (``RAPL'') that Right Adjoints Preserve Limits.

The PAFT is an attempt at a converse to this theorem.
It states that if $G: \mathcal{D} \to \mathcal{C}$ preserves limits, and $\mathcal{D}$ is small and complete, then $G$ is a right adjoint.

The problem with the PAFT is that it's actually a very weak theorem: small complete categories are preorders.

So we want to weaken that requirement of smallness and completeness.
How do we usually weaken smallness?
The next best thing is local smallness, but we're going to lose something if we just replace ``small'' by ``locally small''.

\section{General Adjoint Functor Theorem}
The question is therefore, ``what requirement do we need to add to augment local smallness?''

Recall the most ``concrete'' definition of a left adjoint to $G: \mathcal{D} \to \mathcal{C}$: it is a specification of $F: \mathcal{C} \to \mathcal{D}$ together with $\eta_A: A \to GFA$ for each $A$, such that any $g: A \to GB$ has a unique $h: FA \to B$ with $(Gh) \circ \eta_A = g$.

\newarrow {Dashto} {}{dash}{}{dash}> 

\begin{diagram}
FA & \rDashto^h & B \\
GFA & \rTo^{Gh} & GB \\
\uTo^{\eta_A} & \ruTo~f & {} \\
A & {} & {}
\end{diagram}

This definition is how I remember what an adjoint is, and it very closely parallels the UMP of the free group on a given set.

Now, by TAMO\footnote{\url{http://star.psy.ohio-state.edu/coglab/Miracle.html}}, I define the following Tiny Set Condition:

\

\begin{defn}[Tiny Set Condition] A functor $G: \mathcal{D} \to \mathcal{C}$ has the TSC iff for every object $A \in \mathcal{C}$, there is $B \in \mathcal{D}$ and $\eta_A : A \to GB$ such that every $g: A \to GX$ factors as $A \xrightarrow[\eta_A]{} GB \xrightarrow[Gh]{} GX$ for some $h: B \to X$.

\begin{diagram}
B & \rTo^h & X \\
GB & \rTo^{Gh} & GX \\
\uTo^{\eta_A} & \ruTo~g & {} \\
A & {} & {}
\end{diagram}

\end{defn}

\

Notice how closely this mirrors the definition of an adjoint: it is a very slight weakening of my favourite definition, in that we don't require $h$ to be unique.
In particular, using that definition it is a completely content-free statement that if $G$ has a left adjoint, then it satisfies the TSC: simply take $\eta_A$ to be the unit of the adjunction.

We are assuming access to RAPL, so this gives the following theorem:

\

\begin{thm} A functor $G: \mathcal{D} \to \mathcal{C}$, which has a left adjoint, must satisfy the TSC and must preserve small limits.
\end{thm}

\

\subsection{Relation to the GAFT}

\emph{This is an overview of where we are heading.}

The GAFT talks about what happens if we replace the TSC with a weaker condition:

\

\begin{defn}[Solution Set Condition] A functor $G: \mathcal{D} \to \mathcal{C}$ has the SSC iff for every object $A \in \mathcal{C}$, there is a set $\{ B_i \in \mathcal{D} : i \in I \}$ and $\{ \eta_A^i : A \to G B_i : i \in I \}$ such that all $f: A \to GB$ factor as some $A \xrightarrow[\eta_A^i]{} G B_i \xrightarrow[Gh]{} GB$.
\end{defn}

\

That is, we relax the uniqueness of $B$ in the statement of the TSC.

Then the GAFT states that:

\

\begin{thm}[General adjoint functor theorem] A functor $G: \mathcal{D} \to \mathcal{C}$ (where $\mathcal{D}$ is complete and locally small) has a left adjoint iff it satisfies the SSC and preserves small limits.
\end{thm}

\

\section{Converse to Theorem 1}

We wish to show the following:

\

\begin{thm}[Tiny Adjoint Functor Theorem] If $G: \mathcal{D} \to \mathcal{C}$ has the TSC and preserves small limits, and $\mathcal{D}$ is complete and locally small, then $G$ has a left adjoint.
\end{thm}

\

This will give us a slightly more general version of the PAFT, because we've relaxed smallness of $\mathcal{D}$ and still given an equivalent condition for being an adjoint.

To be clear, what we have done is taken the PAFT, replaced ``small'' with ``locally small'', and imposed the TSC.
The TSC is ostensibly only a slight weakening of the definition of an adjoint, so it's not too much to ask that they would be equivalent.

I will write $FA$ for the object $B$ guaranteed by the TSC; this anticipates the definition of $F$ as a functor.

\subsection{Proof} 
Recall the theorem that a specification of a left adjoint is equivalent to a specification of an initial object of $(A \downarrow G)$ for each object $A \in \mathcal{C}$.
There's one obvious choice for such an initial object: $(FA, \eta_A)$.
(Note for the future that this might not actually be initial, but it's the obvious choice.)
Is this actually initial?

\begin{diagram}
A & {} & {} \\
\dTo^{\eta_A} & \rdTo^f \\
GFA & \rDashto & GX \\
\end{diagram}

Well, it certainly has an arrow from it into any other $(X \in \mathcal{D}, f: A \to GX)$, because that's just the statement of the TSC: any arrow $f: A \to GX$ factors through the map $\eta_A : A \to GFA$.

Is that arrow unique? 
Well, OK, maybe it isn't. The TSC didn't tell us much, after all.
But we are in a complete and locally small category, so what we can do is equalise out all the ways in which the arrow fails to be unique.

\subsubsection{Try and make the arrow unique}
Say $\{ G(h_i): i \in I \}$ is the set of distinct arrows $GFA \to GX$ with $G(h_i) \eta_A = f$.

\begin{diagram}
A & {} & {} \\
\dTo^{\eta_A} & \rdTo^f & {} \\
GFA & \pile{\rTo^{G(h_1)} \\ \rTo_{G(h_2)}} & GX \\ 
\end{diagram}

Let $E$ be the ``industrial-strength equaliser'' of the $h_i$ in $\mathcal{D}$: an arrow $e: E \to FA$ such that $h_i e = h_j e$ for all $i, j$.

Since $G$ preserves limits, $Ge: GE \to GFA$ must be an industrial-strength equaliser of the $G(h_i)$.

\begin{diagram}
{} & {} & A & {} & {} \\
{} & {} & \dTo^{\eta_A} & \rdTo^f & {} \\
GE & \rTo_{Ge} & GFA & \pile{\rTo^{G(h_1)} \\ \rTo_{G(h_2)}} & GX \\ 
\end{diagram}

Since $\eta_A$ equalises all the $G(h_i)$, it must lift uniquely over $Ge$: say $\eta_A = G(e) \circ \overline{\eta_A}$.

\begin{diagram}
{} & {} & A & {} & {} \\
{} & \ldTo^{\overline{\eta_A}} & \dTo^{\eta_A} & \rdTo^f & {} \\
GE & \rTo_{Ge} & GFA & \pile{\rTo^{G(h_1)} \\ \rTo_{G(h_2)}} & GX \\ 
\end{diagram}

Now, in our dream world, the $G(h_i)$ would all be equal: that is, $Ge$ would be an iso.
Can we find an inverse?
The TSC tells us that there is an arrow $G(\gamma) : GFA \to GE$ such that $G(\gamma) \eta_A = \overline{\eta_A}$.
That is, $G(e) G(\gamma) \eta_A = \eta_A$.

\begin{diagram}
{} & {} & A & {} & {} \\
{} & \ldTo^{\overline{\eta_A}} & \dTo^{\eta_A} & \rdTo^f & {} \\
GE & \pile{\rTo^{Ge} \\ \lTo_{G(\gamma)}} & GFA & \pile{\rTo^{G(h_1)} \\ \rTo_{G(h_2)}} & GX \\ 
\end{diagram}

This is as far as we can go, because of the weakness of the TSC, telling us nothing about the arrows whose existence it guarantees.

\subsubsection{Try and produce an object which works}
We see that $G(e) G(\gamma)$ is a map $GFA \to GFA$, and we wanted it to be the identity.

What we could do is equalise out all the maps $GFA \to GFA$, and that would tell us that $\eta_A$ lifted over the equaliser.
This would produce an object which we really hope would actually have the uniqueness property.

Let $G(i): GI \to GFA$ be the industrial strength equaliser of all maps $r_i: FA \to FA$ such that $G(r_i) \eta_A = \eta_A$.

\begin{diagram}
{} & {} & A & {} & {} \\
{} & {} & \dTo_{\eta_A} & \rdTo^{\eta_A} & {} \\
GI & \rTo_{G(i)} & GFA & \pile{\rTo^{G(r_1)} \\ \rTo_{G(r_2)}} & GFA\\
\end{diagram}

Now $G(e \gamma) \eta_A = \eta_A$, so $\eta_A$ lifts over $Gi$ to $\eta_A': A \to GI$.

\begin{diagram}
{} & {} & A & {} & {} \\
{} & \ldTo^{\eta_A'} & \dTo_{\eta_A} & \rdTo^{\eta_A} & {} \\
GI & \rTo_{G(i)} & GFA & \pile{\rTo^{G(r_1)} \\ \rTo_{G(r_2)}} & GFA\\
\end{diagram}

Finally, I claim that this is initial. Indeed, it certainly has maps into every $(GX, f)$, because we can just compose $A$'s TSC-map with $G(i)$.

\begin{diagram}
{} & {} & A & {} & {} & {} & {} \\
{} & \ldTo^{\eta_A'} & \dTo_{\eta_A} & \rdTo^{\eta_A}\rdTo(4,2)^f & {} & {} & {} \\
GI & \rTo_{G(i)} & GFA & \pile{\rTo^{G(r_1)} \\ \rTo_{G(r_2)}} & GFA & \rTo_{G(h)} & GX\\
\end{diagram}

The map is unique: if we had $G(x), G(y): GI \to GX$ such that $G(x)\eta_A' = G(y) \eta_A' = f$, say, then make their equaliser $G(e)$ as above.
$\eta_A'$ lifts over $G(e): GE \to GI$, say to $\eta_A'': A \to GE$.

\begin{diagram}
{} & {} & A & {} & {} \\
{} & \ldTo^{\eta_A''} & \dTo_{\eta_A'} &\rdTo^{f} & {} \\
GE & \rTo_{G(e)} & GI & \pile{\rTo^{G(x)} \\ \rTo_{G(y)}} & GX
\end{diagram}

By the TSC, there is a map $G(\gamma) : GFA \to GE$ with $G(\gamma) \eta_A = \eta_A''$, so $G(i) G(e) G(\gamma) \eta_A = \eta_A$.

\begin{diagram}
{} & {} & {} & {} & {} & {} & A & {} & {} \\
{} & {} & {} & {} & {} & \ldTo(6, 2)^{\eta_A'} \ldTo(4,2)_{\eta_A} \ldTo_{\eta_A''} & \dTo_{\eta_A'} &\rdTo^{f} & {} \\
GI & \rTo_{G(i)} & GFA & \rTo_{G(\gamma)} & GE & \rTo_{G(e)} & GI & \pile{\rTo^{G(x)} \\ \rTo_{G(y)}} & GX
\end{diagram}

Now $G(i)$ is an industrial equaliser of things of this form, so $G(i) G(e) G(\gamma) G(i) = G(i)$; $G(i)$ is monic and so $G(e) G(\gamma) G(i) = 1_{GI}$.

Therefore $G(e)$ is split epic; it's also monic because it's an equaliser, so it's iso.
That is, $G(x), G(y)$ are equal after all, because an equaliser is iso iff the arrows it's equalising are the same.

\subsection{Summary}
We've seen that $G: \mathcal{D} \to \mathcal{C}$ (where $\mathcal{D}$ is complete and locally small) has a left adjoint iff it preserves small limits and has the TSC.

Notice how often in the proof we had lines like ``this thing lifts over this thing'': effectively telling us that the limits we had constructed were genuinely members of $(A \downarrow G)$, so that we could use the TSC.
There is actually a theorem about this, and it follows basically the same pattern as those lines do.
It tells us that limits in the arrow category are basically the same as limits in $\mathcal{D}$.
However, I wanted to stay as concrete as possible here.

\section{More generality}

This condition has told us that ``if you look like you've got a unit of an adjunction, then you really do have a unit''.
That's still quite restrictive, and it turns out to be possible to relax the TSC to the SSC.

Recall that the difference between the TSC and the SSC is just that the SSC asserts the existence of several objects which work, rather than just one.

\

\begin{thm}[GAFT from TAFT] If $G: \mathcal{D} \to \mathcal{C}$ has the SSC and preserves products, and $\mathcal{D}$ has products and is locally small, then $G$ has the TSC.
\end{thm}

\

The proof is by the only way we have of combining objects: that is, by taking products.

If $G$ has the SSC, we construct for each $A$ an object $P = \prod_j B_j$.
Since $G$ preserves products, $GP$ is a product of the $G B_i$.
Define $\eta_A: A \to GP$ componentwise by $\pi_{G B_i} \eta_A = \eta_A^i$.

Now any $f:A \to GB$ factors as $A \xrightarrow[\eta_A^i]{} G B_i \xrightarrow[Gh]{} GB$, because we had a solution set.
Therefore it factors as $A \xrightarrow[\pi_{G B_i} \eta_A]{} G B_i \xrightarrow[Gh]{} GB$; this is already a factorisation through $P$ in the first arrow!
(Of course, the remaining components of $A \to GP$ are defined by anything we like; the solution-set gives us arrows $A \to G B_j$ for $j \not = i$.)

So we are done: given the SSC, we have produced an object $P$ and arrow $\eta_A: A \to GP$ which together witness that $G$ satisfies the TSC.

Combining Theorem 4 with Theorem 3 gives exactly the General Adjoint Functor Theorem.

\end{document}
