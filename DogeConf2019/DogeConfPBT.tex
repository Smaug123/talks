\documentclass{beamer}
\usetheme{Boadilla}
\usepackage{listings}
\usepackage{amsmath}
\usepackage{caption}
\usepackage{fancyvrb}
\usepackage{xcolor}

\usepackage{color}
\definecolor{bluekeywords}{rgb}{0.13,0.13,1}
\definecolor{greencomments}{rgb}{0,0.5,0}
\definecolor{turqusnumbers}{rgb}{0.17,0.57,0.69}
\definecolor{redstrings}{rgb}{0.5,0,0}

\lstdefinelanguage{FSharp}
    {morekeywords={let, new, match, with, rec, open, module, namespace, type, of, member, and, for, in, do, begin, end, fun, function, try, mutable, if, then, else},
     keywordstyle=\color{bluekeywords},
     sensitive=false,
     morecomment=[l][\color{greencomments}]{///},
     morecomment=[l][\color{greencomments}]{//},
     morecomment=[s][\color{greencomments}]{{(*}{*)}},
     morestring=[b]",
     stringstyle=\color{redstrings}
     }

\title{Property-Based Testing}
\author{Patrick Stevens}
\institute{G-Research}
\date{Doge Conf 2019}

\lstnewenvironment{fslisting}
  {
    \lstset{
        language=FSharp,
        basicstyle=\small\ttfamily,
        breaklines=true,
        columns=fullflexible}
  }
  {
  }

\begin{document}

\begin{frame}
\titlepage
\end{frame}

\section{What's the problem?}
\subsection{A program to test}

\begin{frame}
\tableofcontents
\end{frame}

\begin{frame}
\frametitle{Let's have something to test}
Interval set: a space-efficient set of integers

Defining example:
$$\{{\color{red}1,2,3,4},{\color{cyan}8,9,10}\}$$ $$[{\color{red}(1, 4)}, {\color{cyan}(8, 10)}]$$
\end{frame}

\begin{frame}[fragile]
\frametitle{Public API}
\begin{fslisting}
type IntervalSet

[<RequireQualifiedAccess>]
module IntervalSet =
    val empty : IntervalSet
    val add : int -> IntervalSet -> IntervalSet
    val contains : int -> IntervalSet -> bool
\end{fslisting}

\end{frame}

\begin{frame}[fragile]
\frametitle{Data structure}

\begin{fslisting}
type private Interval =
    {
        Least : int
        Greatest : int
    }

type IntervalSet = private IntervalSet of Interval list

\end{fslisting}
\end{frame}

\begin{frame}[fragile]
\frametitle{Implementation: empty}

\begin{fslisting}
[<RequireQualifiedAccess>]
module IntervalSet =
    let empty = IntervalSet []

\end{fslisting}
\end{frame}

\begin{frame}[fragile]
\frametitle{Implementation: insertion}
\begin{fslisting}
let private rec add' (a : int) (ls : Interval list) =
    match ls with
    | [] -> [{ Least = a ; Greatest = a }]
    | interval :: is ->
        if interval.Least <= a && a <= interval.Greatest then
            ls // no need to add, it's already there
        elif interval.Least - 1 = a then
            { interval with Least = interval.Least - 1 }
            :: is // augment this interval to contain a
        elif interval.Greatest + 1 = a then
            { interval with Greatest = interval.Greatest + 1 }
            :: is // augment this interval to contain a
        else
            add' a is // can't add it here; recurse

let add (a : int) (IntervalSet intervals) =
    add' a intervals
    |> IntervalSet
\end{fslisting}
\end{frame}

\begin{frame}[fragile]
\frametitle{Implementation: containment}

\begin{fslisting}
[<RequireQualifiedAccess>]
module IntervalSet =
    let private rec contains' (a : int) (ls : Interval list) =
        match ls with
        | [] -> false
        | interval :: is ->
            if interval.Least <= a && a <= interval.Greatest then
                true
            else
                contains' a is

    let contains (a : int) (IntervalSet intervals) : bool =
        contains' a intervals

\end{fslisting}
\end{frame}

\subsection{Testing the program}

\begin{frame}[fragile]
\frametitle{Start testing!}
Helper function for tests:
\begin{fslisting}
let create (is : int list) : IntervalSet =
    is
    |> List.fold
        (fun set i -> IntervalSet.add i set)
        IntervalSet.empty
\end{fslisting}
\end{frame}
\begin{frame}
\frametitle{What can we test?}
We should test some different lists and their resulting IntervalSets.

\begin{itemize}
    \item $[3; 4]$ contains $5$? (No.)
    \item $[3; 5]$ contains $5$? (Yes.)
    \item $[3; 4; 5]$ contains $4$? (Yes.)
\end{itemize}
\end{frame}

\begin{frame}[fragile]
\frametitle{The test cases}
\begin{fslisting}
    create [3; 4]
    |> IntervalSet.contains 5
    |> shouldEqual false

    create [3; 5]
    |> IntervalSet.contains 5
    |> shouldEqual true

    create [3; 4; 5]
    |> IntervalSet.contains 4
    |> shouldEqual true
\end{fslisting}
Hooray, the tests pass!
\end{frame}

\subsection{But can I really trust myself?}

\begin{frame}
\frametitle{... but is it right?}
\end{frame}

\begin{frame}
\frametitle{... but is it right?}

I don't know!

\begin{itemize}
\item I'm lazy
\item I'm stupid
\item I hate testing
\end{itemize}

Isn't there a better way?
\end{frame}

\begin{frame}[fragile]
\frametitle{FsCheck can help!}

Sneak peek: FsCheck will tell us that this implementation is wrong.

\begin{Verbatim}[commandchars=\\\{\}]
\textcolor{red}{Falsifiable}, after 35 tests (15 shrinks)
  (StGen 9514417537,296661223)):
Original:
[0; 0; 0; 0; 0; 0; 0; 12; 1; -2; 0; 0; 0; 0; 0; 0; 0]
12
Shrunk:
[12; 0]
12
\end{Verbatim}

\end{frame}

\begin{frame}[fragile]
\frametitle{FsCheck's test}
The last two lines FsCheck gave us were:
\begin{verbatim}
[12; 0]
12
\end{verbatim}

FsCheck found this test:

\begin{fslisting}
create [12; 0]
|> IntervalSet.contains 12
|> shouldEqual true
\end{fslisting}
\end{frame}

\section{Introduction to FsCheck}
\subsection{FsCheck's view of the world}

\begin{frame}
\tableofcontents
\end{frame}

\begin{frame}
\frametitle{Why do you test?}

\begin{itemize}
    \item Your program does what you want it to.
    \item Your program doesn't do what you don't want it to.
\end{itemize}
\end{frame}

\begin{frame}
\frametitle{How do you normally test?}

\begin{enumerate}
    \item Come up with examples.
    \item Work out what your program should do on those examples.
    \item Run the program and check it did what you wanted.
\end{enumerate}
\end{frame}

\begin{frame}
\frametitle{But what are you \emph{really} doing?}

You're testing \emph{properties} through \emph{representative examples}.
\end{frame}

\begin{frame}
    Why not just \emph{test properties}?
\end{frame}

\begin{frame}
\frametitle{FsCheck tests properties automatically}

\begin{itemize}
    \item Find edge cases
    \item Find large, complicated cases
    \item Shrink large cases automatically
    \item Make sure you can repeat any failures (the TDD way!)
\end{itemize}
\end{frame}

\subsection{Back to the example}

\begin{frame}
\frametitle{The failing property for IntervalSet}
\begin{itemize}
\item Create an IntervalSet from a list of integers\dots
\item then check for containment\dots
\item should be the same as checking the original list.
\end{itemize}
\end{frame}

\begin{frame}[fragile]
\frametitle{The failing property for IntervalSet, in code}
\begin{fslisting}
let property (ints : int list) (toCheck : int) : bool =
    create ints
    |> IntervalSet.contains toChece
    |> (=) (List.contains doesContain ints)
\end{fslisting}

% Remember, FsCheck told us that this property failed.
\end{frame}

\begin{frame}[fragile]
\frametitle{Invoke FsCheck}
\begin{fslisting}
open FsCheck

let property (ints : int list) (toCheck : int) : bool =
    create ints
    |> IntervalSet.contains toCheck
    |> (=) (List.contains doesContain ints)

[<Test>]
let testProperty () =
    Check.QuickThrowOnFailure property
\end{fslisting}

% Remember, FsCheck told us that this failed.
\end{frame}

\begin{frame}[fragile]
\frametitle{FsCheck's output, revisited}
\begin{Verbatim}[commandchars=\\\{\}]
Falsifiable, after \textcolor{red}{35} tests (15 shrinks)
  (StGen 9514417537,296661223)):
Original:
[0; 0; 0; 0; 0; 0; 0; 12; 1; -2; 0; 0; 0; 0; 0; 0; 0]
12
Shrunk:
\textcolor{cyan}{[12; 0]}
\textcolor{cyan}{12}
\end{Verbatim}

\hfill \break

FsCheck constructed \textcolor{red}{35} tests before finding a failure.
It then shrank the test case to \textcolor{cyan}{the smallest failure} it could find.
\end{frame}

\subsection{Advantages}

\begin{frame}
\frametitle{Advantages}

\begin{itemize}
\item No thought required!
\item Perfectly reproducible
\item Edge cases automatically examined closely
\item Randomised testing increases coverage
\end{itemize}

\hfill \break

(Make sure you explicitly test any failures FsCheck finds, so that nothing is lost to the mists of time!)
\end{frame}

\subsection{What was the bug?}
\begin{frame}[fragile]
\frametitle{The bug}
\begin{fslisting}
let private rec add' (a : int) (ls : Interval list) =
    match ls with
    | [] -> [{ Least = a ; Greatest = a }]
    | interval :: is ->
        if interval.Least <= a && a <= interval.Greatest then
            ls
        elif interval.Least - 1 = a then
            { interval with Least = interval.Least - 1 }
            :: is
        elif interval.Greatest + 1 = a then
            { interval with Greatest = interval.Greatest + 1 }
            :: is
        else
            add' a is // <-- Oh no!
\end{fslisting}
\end{frame}

\begin{frame}[fragile]
\frametitle{The fix}
\begin{fslisting}
let private rec add' (a : int) (ls : Interval list) =
    match ls with
    | [] -> [{ Least = a ; Greatest = a }]
    | interval :: is ->
        if interval.Least <= a && a <= interval.Greatest then
            ls
        elif interval.Least - 1 = a then
            { interval with Least = interval.Least - 1 }
            :: is
        elif interval.Greatest + 1 = a then
            { interval with Greatest = interval.Greatest + 1 }
            :: is
        else
            interval :: add' a is
\end{fslisting}
\end{frame}

\begin{frame}[fragile]
\frametitle{FsCheck is happy}
\begin{verbatim}
Ok, passed 100 tests.
\end{verbatim}
\end{frame}

\begin{frame}[fragile]
\frametitle{Another classic: serialisers}

\begin{verbatim}
[<RequireQualifiedAccess>]
module FancyThing =
    val toString : FancyThing -> string
    val parse : string -> FancyThing option
\end{verbatim}
\end{frame}

\begin{frame}[fragile]
\frametitle{Test the serialiser}

\begin{fslisting}
[<Test>]
let roundTripTest () =
    let property (t : FancyThing) : bool =
        t
        |> FancyThing.toString
        |> FancyThing.parse
        |> (=) (Some t)
    Check.QuickThrowOnFailure property
\end{fslisting}
\end{frame}

\section{Metatesting}
\tableofcontents

\subsection{Was the testing comprehensive?}

\begin{frame}[fragile]
\frametitle{How can we be sure we tested enough?}

Recall the property:
\begin{fslisting}
let property (ints : int list) (toCheck : int) : bool =
    create ints
    |> IntervalSet.contains toCheck
    |> (=) (List.contains doesContain ints)

\end{fslisting}

By fluke (or my incompetence), FsCheck might never generate a ``yes, does contain'' case.
\end{frame}

\begin{frame}
\frametitle{Gather metrics}
F\# is impure and side-effectful, so it's extremely easy to gather metrics.

(In Haskell, this is harder and requires more library functions.)
\end{frame}

\begin{frame}[fragile]
\frametitle{Gather metrics in the property}
\begin{fslisting}
let property
    (positives : int ref)
    (negatives : int ref)
    (ints : int list)
    (toCheck : int)
    : bool
    =
    let contains = List.contains doesContain ints
    if contains then
        incr positives
    else
        incr negatives

    create ints
    |> IntervalSet.contains toCheck
    |> (=) contains
\end{fslisting}
\end{frame}

\begin{frame}[fragile]
\frametitle{Invoke the augmented test}
\begin{fslisting}
[<Test>]
let test () =
    let pos = ref 0
    let neg = ref 0
    Check.QuickThrowOnFailure (property pos neg)

    let pos = pos.Value
    let neg = neg.Value

    pos |> shouldBeGreaterThan 0
    neg |> shouldBeGreaterThan 0

    (float pos) / (float pos + float neg)
    |> shouldBeGreaterThan 0.1
\end{fslisting}
\end{frame}

\begin{frame}
At least a tenth of the time, we want to be hitting positive cases, but \dots
\end{frame}

\begin{frame}[fragile]
\frametitle{The test is a bit unreliable!}
\begin{verbatim}
Expected: 0.1
Actual: 0.08

at FsUnitTyped.TopLevelOperators.shouldBeGreaterThan
\end{verbatim}
\end{frame}

\subsection{Manipulating the cases}
\begin{frame}
\frametitle{Manipulating the generated cases}
We want to generate cases that aren't so often ``look for something that's not in the list''.

FsCheck gives us access to its \emph{generators} for this purpose.
\end{frame}

\begin{frame}[fragile]
\frametitle{Generators}
We will have the property remain the same, but tell FsCheck to generate different cases.

FsCheck has a number of built-in generators.
It also has a computation expression to manipulate generators.
\end{frame}

\begin{frame}[fragile]
\frametitle{Generator of bounded integers}

\begin{fslisting}
let someInts : Gen<int> = Gen.choose (-100, 100)
Gen.sample 0 5 someInts
// output: [57; -24; 67; -14; 77]
\end{fslisting}
\end{frame}

\begin{frame}[fragile]
\frametitle{Generator of bounded even integers}
\begin{fslisting}
let someInts : Gen<int> = Gen.choose (-100, 100)
let evenIntegers : Gen<int> = gen {
    let! (anyInt : int) = someInt
    return 2 * anyInt
}
Gen.sample 0 5 someInts
// output: [-190; -24; -194; -108; -112]
\end{fslisting}
\end{frame}

\begin{frame}[fragile]
\frametitle{Generator that sometimes selects from a list}
\begin{fslisting}
let integers : Gen<int list> =
    Gen.sized (fun i ->
        Gen.choose (-100, 100)
        |> Gen.listOfLength i)

let listAndElt : Gen<int * int list> = gen {
    let! (list : int list) = integers
    let genFromList = Gen.elements list
    let genNotFromList =
        Gen.choose (-100, 100)
        |> Gen.filter (fun i -> not <| List.contains i list)
    let! number = Gen.oneOf [genFromList ; genNotFromList]
    return (number, list)
}
\end{fslisting}
\end{frame}

\begin{frame}[fragile]
\frametitle{Using this generator}

The generator makes pairs of an integer and a list, where the integer is 50\% likely to appear in the list.

\begin{fslisting}
[<Test>]
let test () =
    let pos = ref 0
    let neg = ref 0

    (fun (list, elt) -> property pos neg elt list)
    |> Prop.forAll (Arb.fromGen listAndElt)
    |> Check.QuickThrowOnFailure

    let pos = pos.Value
    let neg = neg.Value

    (float pos.Value) / (float pos.Value + float neg.Value)
    |> shouldBeGreaterThan 0.1
\end{fslisting}
\end{frame}

\begin{frame}[fragile]
\begin{verbatim}
Ok, passed 100 tests.
\end{verbatim}

In fact we now have a positive case about 50\% of the time.
\end{frame}

\begin{frame}
\tableofcontents
\end{frame}

\section{Stateful systems}
\begin{frame}
\frametitle{Stateful systems}
What about state?

Key idea: \emph{describe what to do}, and then do it.
\end{frame}

\subsection{Example}
\begin{frame}[fragile]
\frametitle{Example: a stream}
Simple model: array and pointer.

\begin{table}
\captionsetup{labelformat=empty}
\caption{Starting state, pointer at index $0$}
\begin{tabular}{ |c|c|c|c|c|c|c|c }
    \hline
    \color{red}72 & 69 & 76 & 76 & 79 & 32 & 87 & \dots \\
    \hline
\end{tabular}
\end{table}

\begin{table}
\captionsetup{labelformat=empty}
\caption{Seek to index $2$}
\begin{tabular}{ |c|c|c|c|c|c|c|c }
    \hline
    72 & 69 & \color{red}76 & 76 & 79 & 32 & 87 & \dots \\
    \hline
\end{tabular}
\end{table}

\begin{table}
\captionsetup{labelformat=empty}
\caption{Write $88$ at the current index}
\begin{tabular}{ |c|c|c|c|c|c|c|c }
    \hline
    72 & 69 & \color{red}88 & 76 & 79 & 32 & 87 & \dots \\
    \hline
\end{tabular}
\end{table}

\end{frame}

\begin{frame}[fragile]
Imagine we can't access the implementation, but we want to test it anyway.

\begin{fslisting}
type Stream

[<RequireQualifiedAccess>]
module Stream =
    val uninitialised : unit -> Stream

    val read : Stream -> byte
    val write : Stream -> byte -> unit
    val seek : Stream -> int -> unit
    val currentIndex : Stream -> int
\end{fslisting}
\end{frame}

\begin{frame}
\frametitle{Some things to test}
\begin{itemize}
    \item Write then read
    \item Seek then get index
    \item Write, seek away, seek back, read
\end{itemize}

\hfill \break
FsCheck will do these, and do them well, but they are all quite specific.

Isn't the point of FsCheck to take this drudgery away from us?
\end{frame}

\begin{frame}
\frametitle{Obstacles to FsCheck}

Why is FsCheck not helping here?

\begin{itemize}
    \item Testing a mutable object
    \item No obvious immutable model to use
    \item How to generate random streams?
    \item Need shrinking not to interfere with itself
\end{itemize}
\end{frame}

\begin{frame}
Answer: \emph{describe} what to do, and \emph{then} do it!

\hfill \break

\tiny{c.f. initial algebras}

\end{frame}

\subsection{Testing with FsCheck}

\begin{frame}[fragile]
\frametitle{Domain model}
\begin{fslisting}
type StreamInteraction =
| Write of byte
| Read
| Seek of int
| CurrentIndex

type TestCase = StreamInteraction list
\end{fslisting}

\end{frame}
\begin{frame}[fragile]
\begin{fslisting}
[<RequireQualifiedAccess>]
module TestCase =
    let rec prepareStream
        (s : Stream)
        (instructions : TestCase)
        =
        for instruction in instructions do
            match instructions with
            | Write b, instructions ->
                Stream.write s b
            | Read, instructions ->
                Stream.read s |> ignore
            | Seek n, instructions ->
                Stream.seek s n
            | CurrentIndex, instructions ->
                Stream.currentIndex s |> ignore

\end{fslisting}
\end{frame}

\begin{frame}
\frametitle{Suddenly an immutable model appeared!}

\begin{itemize}
\item By constructing test cases through their \emph{descriptions}\dots
\item we made an immutable model of the world.
\item FsCheck can generate these things completely automatically!
\end{itemize}
\end{frame}

\begin{frame}[fragile]
\frametitle{Immediately useful...}
This can be used with no further modification to check a very useful property:

\begin{fslisting}
[<Test>]
let doesNotCrash () =
    let property (instructions : StreamInteraction list) : bool =
        let s = Stream.uninitialised ()
        TestCase.prepareStream s instructions
        true

    Check.QuickThrowOnFailure (property)
\end{fslisting}
\end{frame}

\begin{frame}[fragile]
\frametitle{... and more useful with generalisation}
... but it really shines with just a little more work.

\begin{fslisting}
    TODO 
\end{fslisting}
\end{frame}

\section{Summary}
\begin{frame}
\frametitle{Why you should use PBT}
Property-based testing...
\begin{itemize}
\item is easy
\item exists in many languages (Python, F\#, Haskell, \dots)
\item can be added incrementally
\item is as comprehensive as you want
\item tests anything (black-box or otherwise)
\end{itemize}
\end{frame}

\end{document}